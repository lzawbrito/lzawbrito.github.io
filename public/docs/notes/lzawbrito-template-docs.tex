\documentclass{article}
\usepackage[usefancyhdr]{/Users/lzawbrito/latex-templates/lzawbrito-template}

\setdocnames{Lucas Z. Brito}{My LaTeX Template}
\begin{document}
\maketitle

\begin{tocbox}
\tableofcontents
\end{tocbox}

\section{Quick Example(s)}
You can use the template by importing with the path to the \texttt{.sty} file 
as the package name (note the lack of a \texttt{.sty} extension in the import 
path):
\begin{minted}{latex}
\documentclass{article}
\usepackage{/path/to/lzawbrito-template} % no .sty
\end{minted}

If you want to use \texttt{fancyhdr}, you can provide the import with the 
\texttt{usefancyhdr} option:
\begin{minted}{latex}
\documentclass{article}
\usepackage[usefancyhdr]{/path/to/lzawbrito-template}
\setdocnames{<author>}{<title>}[<course>] % required!
\end{minted}
In doing so, however, you will be required to 
provide a \mintinline{latex}{\setdocnames} command so that \texttt{fancyhdr} knows what to 
put in the header and LaTeX knows what the document title/author/etc. are. The
reason for this design is it is somewhat cumbersome to (a) set the \texttt{fancyhdr}
contents,\footnote{Especially with the sans-serif font as there is no easy way, as far as I 
know, to change the \texttt{fancyhdr} styling.} and (b) set the 
\mintinline{latex}{\title}, \mintinline{latex}{\author}, etc. of the document to the same text as the 
header without having to provide the same arguments manually. 
To be explicit, the \mintinline{latex}{\setdocnames} turns 
\begin{minted}{latex}
\lhead{\sffamily <author>}
\chead{\sffamily <title>}
\rhead{\sffamily <course>}
\title{\textbf{\textsf{<title>}}}
\author{<author>\\\texttt{<course>}}
\end{minted}
into \mintinline{latex}{\setdocnames{<author>}{<title>}[<course>]}. 

Note that the third argument is optional; if you don't provide it, the document 
title and author will be the only elements of the header, flush fully left and
right as in this document. If a course is provided, the title will be centered
and the course flush right.

% \section{Boxes}

% \section{Equations, Problems, etc.}
% \begin{minted}{latex}
% \begin{equation*}
% 	\eqnmarkbox[black]{a}{\sum_i A_i} + \eqnmarkbox[myred]{b}{B} = C
% \end{equation*}
% \annotate[]{below}{a}{a label}
% \annotate[yshift=1em]{above}{b}{another label}
% \end{minted}
% \vspace{2em}
% \begin{equation}
% 	\eqnmarkbox[myred]{a}{\sum_i A_i} + \eqnmark{b}{B} = C
% \end{equation}
% \annotate[]{below}{a}{a label}
% \annotate[yshift=1em]{above}{b}{another label}

% \begin{equation}
% 	{\color{myblue} A}  + {\color{mygreen} B} + {\color{myred}C}
% \end{equation}

\end{document}