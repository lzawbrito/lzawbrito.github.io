\documentclass{article}
\input{...}

\lhead{Lucas Z. Brito}
\chead{Problem Set 4}
\rhead{PHYS2470}

\title{\textbf{\textsf{Problem Set 4}}}
\author{Lucas Z. Brito\\\texttt{PHYS2470}}

\begin{document}
\maketitle
\problem{Goldenfeld 7-1}
\begin{qbox}
	Consider the Landau free energy 
	\begin{equation*}
		L = \int \dd[d]{x} \qty{ 
			\frac{1}{2}(\del\phi)^2 + \frac{1}{2}r_0\phi^2 
			+ \frac{u_n}{n!}\phi^n
		 }
	\end{equation*}
	\begin{enumerate}[label=\alph*)]
	\item Use the Ginzburg criterion or dimensional analysis to 
		find the upper critical dimension.
	\item Comment on the accuracy of the tricritical exponents 
		which were calculated in exercise 5-2, as a function of dimension.
	\item Show that higher powers of $ \del\phi $ and higher derivatives 
		of $ \phi $ are negligible as $ T\rightarrow T_c $.
	\end{enumerate}
\end{qbox}

\solution
Lorem ipsum dolor sit amet, consectetur adipiscing elit. Phasellus non libero
est. Nunc semper odio lacinia ultrices volutpat. In et dignissim sem. Sed a est
dui. Praesent consectetur sodales suscipit. Donec nisl lacus, elementum quis
auctor eget, luctus a risus. Morbi est augue, interdum nec sem ut, volutpat
pretium dui. Nullam et dui id ipsum egestas fermentum sed ac massa. Mauris
ultrices congue diam sed finibus. Mauris congue ante a felis luctus, ut
imperdiet purus rhoncus. Nunc rhoncus porttitor erat. Maecenas in nisl justo.
Duis dapibus justo magna. Aenean non sapien ultrices, vestibulum massa eget,
feugiat odio. Praesent erat dolor, mollis eu neque id, interdum tempus lorem.
Praesent blandit aliquet dictum.  

\begin{equation}\label{eq:dirac}
(i\slashed{\partial} -m )\underbrace{\cancel{\psi(x)}}_{\text{=1}} = 0
\end{equation}
See equation \eqref{eq:dirac}. 

\end{document}
